\chapter{Änderungen (Code)} \label{chap:AenderungenCode}

\section{App}

\subsection{AsyncPersistor}
Dem Konstruktor muss zusätzlich der Context übergeben werden, damit das Public Key File für das Encryption-Modul geladen werden kann.

\subsection{Encryptor}
Die Methode encrypt() von Encrpytor nimmt nun die Parameter encrypt(File[ ] input, File[ ] output, InputStream publicKey, File encKey). Da nun File Arrays übergeben werden wird erlaubt eine beliebige Anzahl Files mit einem einzigen SecretKey zu verschlüsseln und dadurch eine bessere Erweiterbarkeit erreicht. \par   
Der InputStream ist nötig, da Android nur Klassen mit zugriff auf den Context erlaubt Resourcen zu laden. Dies wirkt sich auch auf IKeyEncryptor aus.

\subsection{Ringbuffer} \label{Ringbuffer}
Uns war von Beginn an klar, dass der Ringbuffer der App, der die Video-Aufzeichnungen der App vor der Persistierung puffert, eine Herausforderung werden würde. Daher haben wir uns bereits in der Entwurfsphase intensiv mit der Umsetzung auseinander gesetzt und uns dafür entschieden, kurze Video-Stücke in einer Warteschlange zu speichern und diese beim Persistieren zusammenzufügen. \par  
Der MediaRecorder, der die Video-Stücke aufnimmt, kann entsprechend parametrisiert werden, so dass er für ein festes Zeitintervall aufnimmt, und dann die Aufnahme beedet. Die Beendigung der Aufnahme wird dem CameraHandler mitgeteilt, worauf hin dieser mit der Aufnahme des nächsten Video-Stücks starten kann. Bei der Implementierung stießen wir jedoch auf das Problem, dass der MediaRecorder asynchron zum Schreiben des Videos Rückmeldung gibt. Somit ist nicht garantiert, dass das Video-File vollständig geschrieben wurde. Aus diesem Grund mussten wir einen Mechanismus entwickeln, um sicherzustellen, dass die Files fertig geschrieben werden, bevor wir versuchen zu persistieren. 
Unser erster Versuch war einen Android-FileObserver auf die Files zu setzen und dadurch Rückmeldung zu erhalten, wenn die Files fertig geschrieben sind. Beim Abrufen der Daten muss dann für jedes File in der Warteschlange gewartet werden, bis es geschrieben wurde. Die Funktion getData() wurde in demandData() umbenannt um klarzustellen, dass eventuell Wartezeit auftreten kann. Es wurden jedoch nicht alle Events empfangen, da manche Files bereits vor dem Einfügen in die Warteschlange vollständig geschreiben wurden.   \par  
Der zweite Ansatz wurde auf einem eigenen Branch entwickelt: Die Gundidee war ein FileLock zu verwenden, welches das Video-File \textit{locken} können sollte, wenn der MediaRecorder den Schreibvorgang beendet hatte. Allerdings wurde kein Lock vom MediaRecoder gehalten, weshalb dieser Ansatz unser Problem ebenfalls nicht löste. \par
Die Problemlösung brachte schließlich user dritter Ansatz: Der FileObserver wurde auf den Ordner mit den Videos gesetzt anstatt auf die Videos selbst. Somit wurde er über alle Änderungen in dem Ordner benachrichtigt, unabhängig davon, wann genau sie gemacht werden.  \par  

Da eine generische Impementierung des RingBuffers den spezifischen Anforderungen der File-Beobachtung nicht mehr gerecht werden konnte, entschieden wir uns ihn zu VideoRingbuffer umzubenennen.  \par  

Für eine erweiterte Funktionalität wurden die Methoden flushAll() und den Inhalt des Puffers zu leeren und destroy() zum Löschen des Puffers hinzugefügt.

\subsection{MemoryManager} \label{MemoryManager}
Die Handhabung temporärer Files wurde erweitert, sodass jede neue MemeoryManager Instanz einen neuen Ordner als aktuellen Ablageort temporärer Daten erzeugt. Dies erleichtert die Arbeit mit den Files, die innerhalb eines Ordner liegen, da garantiert ist, dass der Ordnerinhalt nicht von anderen Threads modifiziert wird. \par
Das Allozieren und Löschen der temporären Files brachte in Android jedoch unerwartet Probleme mit sich. Temporäre Files sollten beim beenden der App gelöscht werden, allerdings wartete unsere App meistens vergeblich auf die Benachrichtigung, der Nutzer hätte die App beendet. Die Folge war eine Anhäufung der temporären Files ohne Chance diese löschen zu können. Daher haben wir die Methode deleteCurrentTempData() und deleteAllTempData() hinzugefügt. \par

\subsection{CameraHandler} \label{CameraHandler}
Da die Implementierung des CameraHandler-Interfaces recht schnell komplex und unübersichtlich werden kann, haben wir Bequemlichkeits-Methoden eingefügt, die einen Lifecycle realisieren und von Klienten in der vorgegebenen Reihenfolge aufgerufen werden. Somit haben Klienten eine bessere Kontrolle über den Handler und können ihn mit ihren eigenen Lifecycles, also dem Lifecycle der Fragments, SurfaceViews oder Activities, synchronisieren. Dieses vorgehen bündelt darüber hinaus kritische Operationen, wie den Kamera-Zugriff, und sensible Anweisungen die zu einer bestimmten Zeit geschehen sollten, wie das löschen der temporären Files, in den neu eingefügten Methoden. Das Ergebnis ist ein klarerer und leichter lesbarer Code.

\section{Web-Dienst}

\subsection{ServerProxy}
Der upload()-Methode wurde ein weiterer Parameter hinzugefügt um Informationen über den Video-Namen weitergeben zu können.  \par  

getVideosByAccount() wurde zu getVideos() umbenannt, da der Parameter accountData den Suffix redundant macht.

\subsection{Main}
Es wurde eine Funktion restartServer() hinzugefügt, falls man zusätzliche Logik einfügen möchte, die nicht der startServer() und stopServer() Routine entspricht (z.B. zum schnelleren Start bei täglichen Neustarts).

\subsection{AccountManager}
Die Methoden setMail() und setPassword() wurden zu changeAccount() zusammengefasst, um die Schnittstelle zu vereinfachen.  \par  

Damit der man der Account Klasse das Salt zum hashen mitgeben kann wurde die Funktion getSalt() hinzugefügt, die das Salt des Accounts aus der Datenbank abfragt.

\subsection{VideoManager} \label{sec:VideoManager}
Da Web-Dienst und Web-Interface zwei vollständig unabhängige Projekte sind ist eine Kommunikation über die Datencontainerklasse VideoInfo nicht direkt möglich. Um dennoch Informationen austauschen zu können wurde die Methode getVideoInfoList() so abgeändert, dass sie nun einen JSON-String erstellt, der versendet und auf dem Interface wieder ausgelesen werden kann.

\subsection{DatabaseManager}
Die Methode getMetaName() wurde hinzugefügt, da der Name gebraucht wird, um das File zu finden. \par

Zudem musste die Methode getSalt() hinzugefügt werden, da am Anfang nicht klar war, dass der Salt in der Datenbank gespeichert werden muss. Zum Entschlüsseln des Passworts, z.B bei der Authentifizierung des Accounts, wird der Salt benötigt, um den Hash zum Vergleichen der Passwörter zu erstellen.

\subsection{Account}
Das lesen des Hash-Salt aus der Datenbank wird vom AccountManager geregelt, da die Account Klasse keinen Zugriff auf die Datenbank hat. Daher war es notwendig die Methode hashPassword() öffentlich zu machen.

\subsection{Metadata}
Die Typen von date (String -> long) und von gForce (Vector3D -> float[]) wurden angepasst und dem Pendant der App zu entsprechen.

\subsection{LocationConfig}
Der LocationConfig mussten weitere Locations für Resources-Ordner hinzugefügt werden, falls man die Ordnerstruktur verändern möchte. \par
Da wir uns entschieden haben einen eigenen Ordner für log-Files zu erstellen wurde auch dieser hinzugefügt.

\subsection{VideoProcessingChain}
Um es bei der Fehlerbehandlung zu ermöglichen einzelne Chains aufzuräumen, wurde eine Funktion cleanUp() hinzugefügt. Diese dupliziert zur Zeit die Funktionalität von deleteTmpFiles(), ist aber nützlich falls man die Fehlerbehandlung erweitern möchte. \par 
Es wurden getter für response und videoName hinzugefügt.

\subsection{VideoProcessing.Chain}
Um die VideoProcessingChain besser zu organisieren wurde das Paket in Unterpakete für die komplexen Bearbeitungsschritte (z.B. Anonymisierung) aufgeteilt

\subsection{OpenCVAnonymizer}
Die Klassen Anonymizer, ExampleAnalyzer und ExampleFilter wurden in OpenCVAnonymizer, OpenCVAnalyzer und OpenCVFilter umbenannt, um aussagekräftigere Namen zu geben.

\subsection{OpenCVFilter}
Der Typ von detections in der methode filter() wurde von Rect in RectVector geändert, da eine andere Version von OpenCV verwendet wird, wie zunächst angenommen und diese einen andere Bezeichnung für die Klasse verwendet. Die Funktionalität bleibt gleich.

\section{Interface}

\subsection{Login}
Die LoginView wird bei jedem Start angezeigt noch bevor der Navigator erzeugt wird. Dadurch wird verhindert, dass durch vor und zurückspringen der Login übergangen werden kann. Zusätzlich hat die MyUI Klasse eine Methode login() bekommen, die den Navigator und das Menü erzeugt und dann zur VideoView wechselt.

\subsection{AccountView}
Die AccountView hat eine Referenz auf die UI bekommen. Somit kann nach Ändern oder Löschen eines Accounts der Benutzer ausgeloggt werden.

\subsection{AccountDataManager}
Die Klasse hatte einen Account als Klassenattribut. Dies wäre zum Problem geworden wenn mehrere Benutzer gleichzeitig eingeloggt sind, da das Attribut ständig überschrieben würde. Deshalb wurde dieses Attribut durch ein von Vaadin bereitgestelltes Sessionattribut, das nur während einem Aufruf der Seite gültig ist, ersetzt.

\subsection{Download} \label{sec:Download}
Der VideoDownload über den, von Vaadin bereitgestellten, FileDownloader stellte uns vor einige Herausforderungen, da dieser verlangte, die herunterzuladende Ressource bereits bei der Initialisierung bereit zu stellen. Zudem muss der erstellte FileDownloader auf einen Button angemeldet werden, der dann den Download startet. \par
Unser erster Versuch war durch einen ersten Button den FileDownloader zu erzeugen und dann durch einen zweiten den Download zu starten. Dabei wurde das Video beim ersten Klick als File geladen. Da der FileDownloader jedoch keine Statusmeldung bereit stellt wussten wir nicht, wann wir das File wieder löschen können. \par
Darauf hin versuchten wir das Video als Stream bereitzustellen. Allerdings hält der FileDownloader den Stream bis zum tatsächlichen Download offen, was dazu führte dass es zu Timeouts kam, wenn man zu lange wartete. \par
Schließlich haben wir einen File-Proxy geschrieben, der gegenüber dem FileDownloader als fertige Ressource agiert, das Video allerdings erst anfragt, wenn der Download tatsächlich gestartet wird, sodass keine Timeouts entstehen. Dies hat zudem erlaubt den ersten Button zu entfernen.

\subsection{Account}
Da wie bei dem \nameref{sec:VideoManager} keine Datenobjekte zwischen Interface und Dienst ausgetauscht werden können, wird auch hier nun ein JSON-String erstellt. Dafür wurde die Methode getAsJson() eingeführt.

\subsection{Video}
Beim Erzeugen der Liste der Videos, werden zuerst die Videos in die Liste eingefügt und anschließend die Metadaten hinzugefügt. Deshalb wird ein Konstruktor in der Klasse Video benötigt, mit dem man ein Video ohne Meta-Informationen erzeugen kann.

\subsection{MailService}
Die Klasse hat eine Methode bekommen, mit der sie E-Mail-Adressen auf korrekte Syntax prüfen kann.